\documentclass[11pt]{article}
\usepackage[a4paper,margin=1in]{geometry}
\usepackage{setspace}
\usepackage{amsmath,amssymb}
\usepackage{graphicx}
\usepackage{hyperref}
\usepackage{natbib}

\onehalfspacing

\title{Lecture Notes: Finance in the Age of Platforms}
\author{E. C. Akoguz}
\date{KU Leuven \\ 2026}

\begin{document}
\maketitle

\section{Introduction}

Over the past two decades, digital innovation has profoundly reshaped the structure of financial markets. Finance is transitioning from a system dominated by vertically and horizontally integrated banks into a modular, platform-based ecosystem in which services are increasingly unbundled, re-bundled, and embedded into broader digital infrastructures.

Finance in the age of platforms is shaped by a tension between modularity and concentration.
While digital technologies lower entry barriers and enable specialization, network effects,
synergies, information spillovers and infrastructure costs continuously drive re-bundling and market power.

\section{Traditional banking before digital innovation}

Before the digital revolution, banking was characterized by large, integrated institutions.
High transaction costs and strong economies of scale and scope favored consolidation and concentration. Banks were characterized by
\begin{itemize}
    \item \textbf{Vertical Integration}: 
Keeping activities in-house helped reducing principal--agent problems, contractual incompleteness, uncertainty and information governance issues.
    \item \textbf{Horizontal Integration}:
Banks expanded horizontally across financial services to exploit economies of scope, cross-selling opportunities, informational spillovers between deposits, payments, credit, and investment services.
    \item \textbf{Market concentration:} High fixed infrastructure and compliance costs, informational barriers to entry, network effects and scale economies resulted in an infertile ground for competition.
\end{itemize}

\section{Digital innovation and the advent of FinTech}

Digital innovation weakened the natural monopoly logic of banking, enabling entry, specialization, and modularity by sharply reducing core transaction costs such as \citep{goldfarb2019digital}:
\begin{itemize}
    \item \textbf{Search and transportation costs}: mobile internet access.
    \item \textbf{Tracking costs}: big data, machine learning, and AI.
    \item \textbf{Verification and compliance costs}: RegTech, digital identity, e-KYC.
    \item \textbf{Infrastructure costs}: cloud computing.
    \item \textbf{Coordination and enforcement costs}: distributed ledger technology (DLT) and smart contracts.
\end{itemize}

This resulted in an unbundling of customer facing financial services, and the back-end services powering them, across specialized actors coordinated through APIs.

\begin{itemize}
    \item E.g.: A consumer can now deposit her money at a digital bank (e.g.\ N26), initiate payments via standalone apps (e.g.\ Payconiq), acces credit from a BNPL business (e.g.\ Klarna) and use an investing platform to manage her portfolio (e.g.\ Robinhood).
    \item E.g. The value chain a payment service is composed of is disintegrated into components such as client-facing wallets (e.g. Revolut, PayPal, Apple Pay), payment initiation service providers (PISPs) (e.g. Tink, Plaid, TrueLayer), processors and gateways (Stripe, Adyen, Worldpay), payment rails (SEPA, Visa, Mastercard, SWIFT), ledger and custody services (by licensed banks) and AML, KYC, fraud detection services by RegTech firms.

\end{itemize}

\section{Re-bundling and platformization of finance}

\paragraph{FinTechs.} Despite the tech-fueled disintegration, re-bundling forces such as
\begin{itemize}
    \item large fixed-cost amortization,
    \item data synergies across services,
    \item user-side and cross-side network effects.
\end{itemize}
remain strong \citep{worldbank2022finance}. Accordingly, major FinTech firms increasingly expand both scale and scope:
\begin{itemize}
    \item Most digital banks and wallets now bundle crypto, insurance, investments, rewards, and bill management.
    \item Stripe evolved from payment processing into lending, fraud prevention, tax handling.
    \item Payconiq merged, integrated, and was acquired into a pan-European payment initiative (EPI).
\end{itemize}
\textbf{Banks.} Banks' business models are gradually transitioning into
\begin{itemize}
    \item \textbf{Banking-as-a-Service (BaaS)}: providing infrastructure and licensing for non-bank digital ecosystems to enable the integration of financial services
    \item \textbf{Banking-as-a-Platform (BaaP)}: serving as the intermediary between customers and third-party services
\end{itemize}
These models allow banks to meet the consumer demand for innovative and affordable digital services all the while leveraging their existing (legal and technical) infrastructure and customer relationships.

\textbf{Public sector.} Digital public infrastructures emerging across the globe aim to facilitate innovation and competition while strengthening digital sovereignty and preserving the efficiency advantages of network and scale effects.
\begin{itemize}
    \item \textbf{Wero:} Introduced by the European Payments Initiative (EPI), Wero is a pan-European digital wallet designed as a modular platform integrating A2A payments, BNPL, digital ID, and loyalty services.
    \item \textbf{Unified Payments Interface:} The Indian instant payment system developed by the National Payments Corporation of India (NPCI) offers inter-bank peer-to-peer (P2P) and person-to-merchant (P2M) transactions as well as extensive add-on services such as autopay for recurring bills, account aggregation, overdraft facilities and international remittances.
    \item \textbf{Pix:} An instant payment system created by the Central Bank of Brazil, Pix enables 24/7, real-time P2P and P2M money transfers as well as services like scheduled payments, BNPL, cash winthdrawal and cashback.
\end{itemize}

\section{Mobile Money.} Mobile money, a financial solution that emerged and scaled in Sub-Saharan Africa,  enables users to store and transfer value via mobile phones without bank accounts. The key drivers behind mobile money's success are
\begin{itemize}
    \item \textbf{network effects}: a large existing base of telecom users,
    \item \textbf{infrastructure advantage}: scalable transaction infrastructure, extensive agent networks,
    \item \textbf{user trust}: acquaintance with topping up phones and trust with the operator with digital value.
    \item \textbf{regulatory flexibility}: more agility and speed compared to heavily regulated banks
    \item \textbf{financial underreach}: a large unbanked or underserved population by traditional financial institutions.
\end{itemize}

The evolution of mobile money is a testimony to the power of network and data effects: It took 18 years for the mobile money industry to reach one billion accounts, but only five years to double that number \citep{gsma2025}. 
In line with the re-bundling trend, several mobile money providers (e.g. Paytm (India), Grab (Singapore), M-Pesa (SSA)) are increasingly expanding their range of service into other financial products in partnership with banks and
FinTechs (GSMA).

\section{Super apps}

Super apps bundle financial and non-financial services - such as deliveries, ticket booking, shopping, ridehailing - forming closed ecosystems where data and activity remain within the platform. 
Thus, superapp users exchange convenience for surveillance. 

The super app model first emerged and scaled in Asia (with WeChat and AliPay), later spreading to parts of Africa and Latin America. Currently there are no super apps at the scale of WeChat or AliPay in western countries. Unlike most Asian countries, western economies are characterized by more fragmented markets, stronger privacy norms, stricter competition and data regulation as well as stronger legacy alternatives for financial services, making them less habitable environments for super apps. 

That being said, the recent initiatives from BigTech companies signal ambition and competition towards building the first wester super app:
\begin{itemize}
    \item \textbf{Meta Pay:} Meta’s unified digital wallet embedded across Facebook, Instagram, Messenger, and WhatsApp, designed to handle peer-to-peer payments, in-app purchases, creator monetization, and eventually broader commerce and financial services. While not enough to constitute a super app by itself, Meta Pay is regarded as a payment spine that could support deeper re-bundling if regulatory conditions allow.
    \item \textbf{X Money:} X’s emerging payments and financial layer, built around a digital wallet and peer-to-peer transfers. X Money currently sits atop existing financial rails and faces tight regulatory constraints. Still, by embedding payments directly into social interaction and identity, X Money represents a step closer towards the super app model, an ambition frequently voiced by the company CEO himself, Elon Musk \cite{Guardian_2023}
\end{itemize}

\section{Big Tech's Entry into Finance}

Several factors motivate the entry and success of Big Tech in financial services: 
\begin{itemize}
    \item \textbf{Data--Network--Activity Feedback Loop}: Big Tech generates large troves of data thanks to network externalities. This data enhances service quality and attracts even more users and more data which allows Big Tech to offer a wider range of activities, yielding even more data \citep{doerr2023big}.
    \item \textbf{Regulatory leeway}: lighter or fragmented oversight allows regulatory arbitrage \cite{buchak2018fintech}.
    \item \textbf{Unmet credit needs}: especially in emerging markets; high banking margins (low competition) create room for entry.
\end{itemize}

\paragraph{Big Tech \& Payments:} Large user networks, rich behavioral data and economies of scope across financial and non-financial services provide a competitive advantage to Big Tech in payment markets, sharply lowering adoption and coordination costs. The dominance of Big Tech in payments in China demonstrates how these factors can result in a winner-takes-all dynamics in the absence of strong regulatory safeguards.

\paragraph{Big Tech \& Lending:}
Big Tech has certain advantages in offering credit such as:
\begin{itemize}
    \item superior monitoring and enforcement \& reduced reliance on collateral
    \item alternative data-based credit scoring,
    \item threat of platform exclusion.
\end{itemize}
Evidence shows that Big Tech lending can improve credit access for constrained borrowers:
\begin{itemize}
    \item Mercado Libre's proprietary credit score can correctly lend to 30\% of borrowers that credit bureaus label as high risk \cite{frost2019bigtech}
    \item In China, SMEs using QR-code payments have up to 87 \% probability of receiving Big Tech credit after 3 years \cite{beck2022big}
    \item Platform lending may improve credit access for the most constrained borrowers even when it is technologically inferior to bank lending \cite{bouvard2022lending}
\end{itemize}

\section{A new source of systemic risk: Cloud 
Computing}

While technological innovation lowers entry barriers and infrastructure costs by enabling the outsourcing of services to specialized providers, it also introduces new concentration risks. Cloud computing is one such example: Cloud computing provides remote and on-demand access to computing power, data storage and software
instead of relying on in-house IT infrastructure.
Over the past decade, financial institutions have
aggressively migrated core systems and data infrastructure to the cloud, seeking to improve their speed, agility, and resilience. Major banks like HSBC, JPMorgan Chase, Standard Chartered, and Barclays are adopting cloud-first strategies. 
Benefits of cloud computing include
\begin{itemize}
    \item overcoming economies of scope by
\begin{itemize}
    \item providing access to cutting-edge technology and analytics that were previously affordable only to large incumbents
    \item allowing businesses to scale resources flexibly with demand and avoid high fixed infrastructure costs
    \item providing built-in compliance frameworks that reduce the regulatory burden on small businesses 
\end{itemize}
\item leveraging network \& ecosystem benefits 
\begin{itemize}
    \item the provider's ecosystem (e.g. marketplaces, data
exchanges, sandboxes) lower coordination and
integration costs, connecting entrants to potential partners
and customers.
\end{itemize}
\end{itemize}
That being said, cloud markets exhibit strong concentration, implying a significant systemic in case of failure of one of the key providers:
\begin{itemize}
    \item Amazon Warehouse Services (AWS) hold 30\% of the global cloud market share by itself, powering over 90\% of Fortune 100 firms.
    \item AWS is followed closely by Azure (20\%) and Google Cloud (10\%)
    \item Critical financial and regulatory infrastructures also rely on these major cloud services for their operations
\end{itemize}
Concentration in the cloud-market is driven by the usual factors such as
\begin{itemize}
    \item \textbf{Ecosystem Network effects:} More users attract more third-party tools, integrations, compatible software, skilled developers, attracting even more users. 
    \item \textbf{Data and Learning Effects:} More clients generate more operational data, helping providers optimize services and attract more clients.
    \item \textbf{Multi-Service Bundling:} Complementary services (analytics, cybersecurity, ML tools) raise value for clients
    \item \textbf{Economies of Scale:} Service costs decrease as the provider grows. 
    \item \textbf{Switching costs/lock-in:} Human capital \& complementary capital that are not perfectly transferable
\end{itemize}

On October 20 2025, AWS reported significant service disruption lasting approximately 15 hours. Coinbase, Robinhood, Venmo, Lloyds Bank, and Bank of Scotland, among many others, experienced major service disruptions. The 2025 AWS outage heightened awareness among policymakers regarding
\begin{itemize}
    \item single-point-of-failure risk,
    \item digital and financial sovereignty concerns,
    \item the emergence of ``too-connected-to-fail'' cloud providers.
\end{itemize}
fuelled by the advent of cloud-computing. Accordingly, regulatory forces are driving the
industry towards more resilient, multi-cloud
solutions (e.g., DORA in the EU,
SS2/21 in the UK).).

\newpage

\bibliographystyle{plain}
\bibliography{references_platforms
}

\end{document}
